\documentclass{article}
\usepackage[a4paper, margin=0.5in]{geometry}
\usepackage[utf8]{inputenc}
\usepackage{amsmath}
\usepackage{stmaryrd}
\usepackage{graphicx}
\usepackage{xcolor}
\usepackage[shortlabels]{enumitem}

\renewcommand\t[1]{\text{#1}}
\renewcommand\d{\text{ . }}
\newcommand\lwt[1]{$_\text{#1}$}
\newcommand\TODO[1]{\textcolor{red}{TODO: #1}}

\setlength\parindent{0pt}

\begin{document}

% \maketitle

\section{}

\begin{gather*}
U = \{\t{mike}, \t{will}, \t{elli}\} \\
\t{atomic prep.} = \{\t{ride\_bike}(x)| x \in U\} \cup \{\t{sleep}(x)| x \in U\} \cup \{\t{tease}(x,y)| x,y \in U\times U\} = \\
\bigg\{
\t{ride\_bike}(\t{mike}), \t{ride\_bike}(\t{will}), \t{ride\_bike}(\t{elli}),
\t{sleep}(\t{mike}), \t{sleep}(\t{will}), \t{sleep}(\t{elli}),
\t{tease}(\t{mike}, \t{mike}), \t{tease}(\t{mike}, \t{will}),\\
\t{tease}(\t{will}, \t{mike}), \t{tease}(\t{will}, \t{will}),
\t{tease}(\t{will}, \t{elli}), \t{tease}(\t{elli}, \t{will}),
\t{tease}(\t{elli}, \t{elli}), \t{tease}(\t{elli}, \t{mike}),
\t{tease}(\t{mike}, \t{elli}) \bigg\}
\end{gather*}

\section{}

Using the preposition ordering from the above.

\begin{gather*}
M = \bigg\{ \\
(
1, % ride mike
0, % ride will
0, % ride elli
0, % sleep mike
0, % sleep will
1, % sleep elli
0, % tease mike, mike
1, % tease mike, will
0, % tease will, mike
0, % tease will, will
0, % tease will, elli
0, % tease elli, will
0, % tease elli, elli
0, % tease elli, mike
0  % tease mike, elli
),\\
(
1, % ride mike
1, % ride will
0, % ride elli
0, % sleep mike
0, % sleep will
1, % sleep elli
0, % tease mike, mike
1, % tease mike, will
0, % tease will, mike
0, % tease will, will
0, % tease will, elli
0, % tease elli, will
0, % tease elli, elli
0, % tease elli, mike
1  % tease mike, elli
),\\
(
0, % ride mike
0, % ride will
0, % ride elli
1, % sleep mike
1, % sleep will
1, % sleep elli
0, % tease mike, mike
0, % tease mike, will
0, % tease will, mike
0, % tease will, will
0, % tease will, elli
0, % tease elli, will
0, % tease elli, elli
0, % tease elli, mike
0  % tease mike, elli
),\\
(
0, % ride mike
0, % ride will
0, % ride elli
1, % sleep mike
0, % sleep will
1, % sleep elli
0, % tease mike, mike
0, % tease mike, will
0, % tease will, mike
0, % tease will, will
0, % tease will, elli
0, % tease elli, will
0, % tease elli, elli
0, % tease elli, mike
0  % tease mike, elli
),\\
(
1, % ride mike
1, % ride will
0, % ride elli
0, % sleep mike
0, % sleep will
1, % sleep elli
0, % tease mike, mike
0, % tease mike, will
0, % tease will, mike
0, % tease will, will
0, % tease will, elli
0, % tease elli, will
0, % tease elli, elli
0, % tease elli, mike
0  % tease mike, elli
),\\
\bigg\}
\end{gather*}

\section{}

\subsection{}

This probability/cuount should be 0. And indeed, in no model does anyone sleep and ride at the same time $\rightarrow 0$.

\subsection{}

Mike rides in 3 models and sleeps in 2 of them. From that, $\frac{3}{5} > \frac{2}{5}$.

However, I would disagree with using the frequency of an action as a proxy for liking something. E.g. in every model the boys brush their teeth but that does not mean that they enjoy it more than riding a bike.

\subsection{}

Elli sleeps in 5 models and rides the bike in 0 of them. From that, $\frac{5}{5} > \frac{0}{5}$.

\subsection{}

Mike teases Will in two models and Elli in no model. From that, $\frac{2}{5} > \frac{0}{5}$.

\subsection{}

Will never teases anybody, therefore the implication is trivially fulfilled.

\subsection*{Further inferences}

\begin{itemize}
\item Nobody can tease while asleep.
\item Nobody teases themselves.
\item Elli doesn't tease anyone and is not teased by anyone.
\end{itemize}

\clearpage

\section{}

\begin{enumerate}[a)]
\item $(1,1,0,0,0)^T$
\item $(1,1,0,0,1)^T$
\item $(0,0,0,0,0)^T$
\item $(0,0,0,0,0)^T$
\item $(0,0,1,1,0)^T$
\end{enumerate}

\section{}

(a) \textit{Mike is sleeping}, (b) \textit{A boy is sleeping}.

\bigskip

$p(a) = \frac{1}{5}, p(a|b) = \frac{1}{2}$, 
$\text{inference}(a,b) = \frac{\frac{1}{2}-\frac{1}{5}}{1-\frac{1}{5}} = \frac{\frac{3}{10}}{\frac{8}{10}} = \frac{3}{8}$

\bigskip

The score is higher than $0$, therefore is a boy sleeps, then it suggests that Mike is also sleeping. However, it is not $1$, therefore there are some cases in which some boy is sleeping but not Mike. With more samples the inference would probably be higher.

\end{document}
