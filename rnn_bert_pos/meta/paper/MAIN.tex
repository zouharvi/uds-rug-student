%
% File acl2020.tex
%
%% Based on the style files for ACL 2020, which were
%% Based on the style files for ACL 2018, NAACL 2018/19, which were
%% Based on the style files for ACL-2015, with some improvements
%%  taken from the NAACL-2016 style
%% Based on the style files for ACL-2014, which were, in turn,
%% based on ACL-2013, ACL-2012, ACL-2011, ACL-2010, ACL-IJCNLP-2009,
%% EACL-2009, IJCNLP-2008...
%% Based on the style files for EACL 2006 by 
%%e.agirre@ehu.es or Sergi.Balari@uab.es
%% and that of ACL 08 by Joakim Nivre and Noah Smith

\documentclass[11pt,a4paper]{article}
\usepackage[hyperref]{acl2020}
\usepackage{times}
\usepackage{latexsym}
\usepackage{xcolor}
\usepackage{amsmath}
\usepackage{booktabs}
\usepackage{cleveref}
\usepackage{enumitem}
\renewcommand{\UrlFont}{\ttfamily\small}
\newcommand{\TODO}[1]{\textcolor{red}{TODO: #1}}
\newcommand{\dagg}{$\dagger{}$}

% This is not strictly necessary, and may be commented out,
% but it will improve the layout of the manuscript,
% and will typically save some space.
\usepackage{microtype}

\aclfinalcopy % Uncomment this line for the final submission
%\def\aclpaperid{***} %  Enter the acl Paper ID here

%\setlength\titlebox{5cm}
% You can expand the titlebox if you need extra space
% to show all the authors. Please do not make the titlebox
% smaller than 5cm (the original size); we will check this
% in the camera-ready version and ask you to change it back.

\title{Hyperparameters of RNN Architectures for POS Tagging using Surface-Level BERT Embeddings}

\author{Vilém Zouhar \\
%  Affiliation / Address line 1 \\
%  Affiliation / Address line 2 \\
%  Affiliation / Address line 3 \\
  \texttt{vzouhar@lsv.uni-saarland.de}
}

\date{}

\begin{document}
\maketitle

\begin{abstract}
    Contextual embeddings from pretrained BERT models have been useful in a variety of natural language processing tasks. This paper focuses on one of the basic of those tasks, Part of Speech tagging, and compares several simple recurrent neural network models (vanilla RNN, GRU, LSTM) and their hyperparameters (hidden state size, recurrent layers and dropout, bidirectional, dense layers).
    
    While stacking recurrent layers or densely connected output layers negatively affects the performance, adding bidirectionality and increasing hidden state size improve it significantly.
\end{abstract}

\section{Introduction}

In comparison to one-hot word embeddings, more complex word embeddings offer significant boosts in performance. One of the main disadvantages of one-hot word embeddings is their high dimensionality and sparsity. Trained word embeddings are able to capture word meaning based on its collocations. The main families of word embeddings are (1) non-contextual, e.g. word2vec \cite{word2vec} and (2) contextual, e.g. BERT \cite{bert}. While the first assigns one word the same embedding regardless of the context, the second one produces different ones based on the context. This improvement comes at the cost of significantly higher training requirements. In this experiment, we make use of pretrained BERT$_\text{BASE}$-Cased made available by \citet{huggingface}.

Part of Speech tagging is the task of assigning part of speech tags to every word in a sentence. We make use of the OntoNotes 4 corpus \cite{ontonotes} of 70k English sentences with POS annotations (49 POS classes).

The goal of this paper is to document the effect of basic recurrent neural network architectures on task performance. The used models: vanilla RNN \cite{rnn} - henceforth only ``RNN'', LSTM \cite{lstm} and GRU \cite{gru} are combined with various architecture hyperparameter choices: hidden state size, recurrent layers and dropout, bidirectionality \cite{bidir} and dense layers.

\section{Methods}

The corpus was split on sentence boundaries to $80\%$ training, $10\%$ development and $10\%$ test sets. The whole batch was used to compute the gradient at each step. The optimizer was Adam \cite{adam} with the learning rate specified with every architecture type. The number of epochs was fixed to $100$, though almost all models would saturate before epoch $25$. Finally, the output of every model is processed by softmax to get a distribution.

Model state with the best performance on the development set was used to evaluate on the test set. We report model performance (by class accuracy) on the test set as well as on the training set (with dropout off). The motivation for the latter is to observe overfitting.

\paragraph{Embeddings}

Vector embeddings for every subword unit were extracted from the top-level (last) layer (768-dimensional vector). Further performance improvements could be made by, e.g. concatenating the top four layers or averaging them.
Token embeddings are computed as the average of corresponding subword embeddings.\footnote{
The embeddings were stored with 16bit float precision to reduce memory footprint.}

\paragraph{Models}

We consider four baselines based on densely connected layers to gain an insight into the performance of models based on single token embedding. The purpose of these configurations is to offer a fast, viable baseline in comparison to more complex models.

Although the embeddings provided by BERT are contextualized, they are frozen in this experiment and may not contain enough information about the rest of the sentence. Recurrent neural networks may be used to alleviate this issue specifically. We experiment with three basic recurrent cells: RNN, LSTM, GRU. For each of this cell, we also explore the following six architecture hyperparameters.

Due to computational limitations, the search for optimal hyperparameters is \textit{not} performed using gridsearch. For every parameter type, we modify only this dimension and keep the rest from the baseline (type+activation, hidden state size, recurrent layers + dropout, bidirectional, dense layers) of (LSTM, 128, 1+0.0, no, 1). We make a conscious faulty assumption regarding the relative independence of these parameters. Despite that, the exploration can still provide us with insight with respect to the generally good architecture hyperparameters.
For brevity, we describe every model and hyperparameters in the respective part of the results.

\section{Results}

\paragraph{Baselines} 
There are four model configurations that utilize densely connected layers (some with dropout added). The activation function for nonfinal layers is Tanh. Densely connected layer with this activation function, $n$ output size and dropout of probability $p$ is denoted as $L_n^p$. We define model versions with an without dropout. The output is then fed through softmax to get a distribution. Batch size is fixed to $2048$ \textit{tokens}; learning rate is $0.001$.
For completeness, we also list the majority class classifier to provide the lowest possible, uninformed baseline.

\begin{table}[ht]
    \centering
    \begin{tabular}{lcc}
    \toprule
    Model & Train & Test \\
    \midrule
    Majority & $13.7\%$ & $13.7\%$ \\
    $L_{49}$\dag{} & $91.7\%$ & $91.4\%$ \\
    $L_{128} \times L_{49}$ & $91.5\%$ & $90.7\%$ \\
    $L_{128}^{0.1} \times L_{49}$ & $91.3\%$ & $90.6\%$ \\
    $L_{256} \times L_{256} \times L_{49}$ & $91.3\%$ & $90.7\%$ \\
    $L_{256}^{0.1} \times L_{256}^{0.1} \times L_{49}$ & $91.3\%$ & $90.8\%$ \\
    \bottomrule
    \end{tabular}
    \caption{\label{tab:baselines} Performance of densely connected baselines}
\end{table}

Baseline results can be found in \Cref{tab:baselines}. The best model (on test) is marked with \dagg. Overfitting does not seem to be an issue, though adding dropout shrinks the gap between train and test performance. Despite that, deeper networks perform slightly worse than just logistic regression. This may be the result of choices of batch size, learning rate and the activation function.\footnote{Leaky ReLU with $\alpha=0.01$ performed even worse.} Furthermore, taking BERT embeddings from, e.g. the top four layers may benefit deeper networks. Even logistic regression provides a very performant baseline, also considering the uninformed majority classifier.

\paragraph{Recurrent Units}
We explore the capabilities of LSTM and GRU units. Furthermore, we may choose different activation functions for the RNN: ReLU and Tanh. Batch size is fixed to $16$ \textit{sentences}, corresponding on average to $348$ tokens; learning rate is $0.0005$. When reporting performance, the base parameter setup is marked with *.

\begin{table}[ht]
    \centering
    \begin{tabular}{lcc}
    \toprule
    Recurrent Unit & Train & Test \\
    \midrule
    RNN+ReLU & $94.2\%$ & $92.9\%$ \\
    RNN+Tanh & $96.3\%$ & $94.8\%$ \\
    LSTM*\dagg & $97.1\%$ & $95.2\%$ \\
    GRU & $96.7\%$ & $95.0\%$ \\
    \bottomrule
    \end{tabular}
    \caption{\label{tab:units} Performance of models recurrent neural networks with different recurrent units}
\end{table}

\Cref{tab:units} shows the performance of different RNN units. All of them outperformed the baseline by a significant margin. Although they all share similar performance, RNN with ReLU activation performed the worst, while LSTM showed to be the most useful for this task.

\paragraph{Hidden State Size} 
A common parameter to modify recurrent neural network capacity is the size of the hidden state vector passed between steps.

\begin{table}[ht]
    \centering
    \begin{tabular}{lcc}
    \toprule
    Hidden state size & Train & Test \\
    \midrule
    16 &   $93.3\%$ & $92.4\%$ \\
    32 &   $94.9\%$ & $93.6\%$ \\
    64 &   $95.5\%$ & $93.8\%$ \\
    128* & $97.1\%$ & $95.2\%$ \\
    256 &  $97.6\%$ & $95.5\%$ \\
    512 &  $97.7\%$ & $95.6\%$ \\
    1024\dagg & $98.1\%$ & $95.9\%$ \\
    \bottomrule
    \end{tabular}
    \caption{\label{tab:hidden_size} Performance of recurrent neural networks with different hidden state size}
\end{table}

The steady increase in performance on both train and test data sets suggest that the model is able to make use of the larger capacity. It is also interesting to see that even with a hidden state size of $16$, the recurrent model is able to outperform the baseline.

\paragraph{Number of stacked Recurrent Layers}

Recurrent neural networks can be stacked together. When larger than $1$, the output of the previous layer is processed by another recurrent network layer. It is also possible to introduce a dropout layer in between these stacked recurrent layers. We also show the results with dropout probability $0.1$.

\begin{table}[ht]
    \centering
    \begin{tabular}{lcc}
    \toprule
    Number of layers & Train & Test \\
    \midrule
    1 layer*\dag{} & $97.1\%$ & $95.2\%$ \\
    2 layers & $95.5\%$ & $93.9\%$ \\
    2 layers + dropout & $95.5\%$ & $94.0\%$ \\
    3 layers & $95.0\%$ & $93.7\%$ \\
    3 layers + dropout & $95.2\%$ & $94.0\%$ \\
    \bottomrule
    \end{tabular}
    \caption{\label{tab:rnn_layers} Performance of recurrent neural networks with different number of recurrent layers stacked}
\end{table}

As documented in \Cref{tab:rnn_layers}, stacking multiple recurrent neural layers worsens the performance in this case. This might be the result of choosing the base hidden state size of $128$ and could perhaps lead to improvements if larger values were used.

\paragraph{Bidirectionality}

If bidirectionality is added, the sentence is processed left to right by one recurrent network and right to left by another one. The final vectors are then concatenated, resulting in twice as large a final vector for every token.

\begin{table}[ht]
    \centering
    \begin{tabular}{lcc}
    \toprule
    Bidirectional & Train & Test \\
    \midrule
    No* & $97.1\%$ & $95.2\%$ \\
    Yes\dagg & $98.4\%$ & $96.7\%$ \\
    \bottomrule
    \end{tabular}
    \caption{\label{tab:bidir} Performance of recurrent neural networks with either bidirectional processing on or off}
\end{table}

Adding bidirectional processing (shown in \Cref{tab:bidir}) vastly improves the performance compared to other architecture hyperparameter choices. A fair comparison would be to a model with a hidden state vector of size $256$ since bidirectional processing uses two hidden state vectors sized $128$ each. Bidirectionality helps even in this case.

\paragraph{Dense Layers}

The output for every token is the hidden state vector at that specific timestep. This can be further processed by deeper, densely connected layers. We again use Tanh as the activation function.

\begin{table}[ht]
    \centering
    \begin{tabular}{lcc}
    \toprule
    Output dense & Train & Test \\
    \midrule
    $L_{49}$*\dagg & $97.1\%$ & $95.2\%$ \\
    $L_{128}\times L_{49}$ & $95.8\%$ & $94.2\%$ \\
    $L_{256}\times L_{256}\times L_{49}$ & $94.7\%$ & $93.5\%$ \\
    \bottomrule
    \end{tabular}
    \caption{\label{tab:rnn_dense} Performance of recurrent neural networks with different shapes of output densely connected layers}
\end{table}

Results in \Cref{tab:rnn_dense} document, that similarly to densely connected networks, deeper architectures only decrease the performance. Different activation functions could lead to different results.

\paragraph{Combination of Best Parameters}

Finally, we train a system with the (individually) best-performing parameters: LSTM, hidden state size $512$,\footnote{Combined hidden state vectors are $1024$-dimensional.} one recurrent network layer, bidirectionality, single output layer. This resulted in an accuracy of $98.9\%$ and $97.1\%$ on training and test sets, respectively. Keeping bidirectionality and increasing hidden state size to $1024$ leads to accuracies of $98.8\%$ and $97.2\%$, respectively.

\section{Conclusion}

In this paper we focused on the six most important hyperparameter choices of simple recurrent neural networks in the task of POS tagging with top-layer BERT embedding.

Out of the four basic recurrent units, LSTM performed the best, although only by a small margin over GRU. We find that deeper architectures are not helpful on the given corpus, not even for the baselines. The two most important hyperparameters which positively affected the performance were bidirectionality and greater hidden state size.

\subsection{Future Work}

This paper made a conscious faulty independence assumption of the inspected parameters. To make this research formally sound, a full gridsearch or a more informed search should be performed.

Furthermore, this paper did not touch upon the batch size, optimizer parameters and different BERT embedding extraction, which also interact with other hyperparameters.

The length limit of this paper also did not allow for error analysis, which could reveal low performance on specific classes and could suggest improvements in this regard.

\bibliography{bibliography}
\bibliographystyle{acl_natbib}

%\appendix

\end{document}
