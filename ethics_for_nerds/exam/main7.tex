\documentclass[a4paper, 11pt]{article}
\usepackage[utf8]{inputenc}
\usepackage[T1]{fontenc}
\usepackage[left=2cm,right=5cm,top=2cm,bottom=2cm]{geometry}
\usepackage[english]{babel}
\usepackage{graphicx}
\usepackage{datetime}
\usepackage{enumerate}
\usepackage{xcolor}
\usepackage{amsmath}
\usepackage[onehalfspacing]{setspace}
\setlength{\parindent}{0em}
\setlength{\parskip}{1em}
\newcommand{\stdn}{} \newcommand{\studentname}[1]{\renewcommand{\stdn}{#1}}
\newcommand{\mtrk}{} \newcommand{\matriculationno}[1]{\renewcommand{\mtrk}{#1}}
\newcommand{\isnr}{} \newcommand{\issuenumber}[1]{\renewcommand{\isnr}{#1}}
\newcommand{\makeheader}{\textbf{Ethics for Nerds} Exam \hfill \textbf{Student name:} \stdn \\ \today, \currenttime \hfill \textbf{Matriculation no.:} \mtrk \par \textbf{\Large Issue \isnr}}

\newcommand{\TODO}[1]{\textcolor{red}{TODO: #1}}

\begin{document}
\sffamily

%PLEASE DON'T CHANGE ANYTHING ABOVE THIS POINT

\studentname{Vilém Zouhar} %FILL THIS IN WITH YOUR NAME
\matriculationno{} %FILL THIS IN WITH YOUR MATRICULATION NO.
\issuenumber{7} %FILL THIS IN WITH THE NUMBER OF THE ISSUE
\makeheader %do not delete this line

%%%%%%%%%%%%%%%%%%%%%%%%%% ANSWER THE ISSUE HERE %%%%%%%%%%%%%%%%%%%%%%%%%%%%%%

(Overleaf uses GMT0 while we are in GMT+2)
\vspace{-0.4cm}

\begin{enumerate}[a)]
    \item
    \begin{itemize}
    \item Privacy:
    The app would contain a lot of personal information about the children.
    They would also be connected to their parents' accounts who could (presumably) monitor the progress as well.
    Even if children were given a choice whether to participate in this, they may be still too young to understand how problematic this is privacy-wise.
    They could also be manipulated by in-app gamification that targets children.
    
    \item Childcare neglect/automation:
    Instead of the parents taking their time to talk with their children, they outsource this duty to a monitoring app.
    Even though it could provide short-term benefits, it could also gradually change the society to become an automation dystopia.
    This poses a risk when this form of communication about mental health issues would be presented to children from their early age when they're susceptible to outside influences.
    
    \item Discrimination:
    Assuming that the app would work as advertised, it could provide a disproportional advantage to families who can afford phones and this service for their children.
    Currently, the availability of phones for children and paid services is limited and thus children from higher societal classes would get even more advantages with regards to mental health issues.
    Because mental health issues are a very serious thing with effects lasting the whole life, this could be considered morally problematic.
    \end{itemize}

    \item
    \begin{itemize}
    \item Unfair competition:
    The system made by the insurance company may unfairly suggest medical professionals who pay them extra provisions to be highlighted on top of the list.
    This would lead to unfair competition and sometimes even faulty advice could be given in case someone who is less experienced to treat a specific mental health issue would be preferred over someone else just because of the way it is presented in the app.
    
    \item Monopolization:
    The insurance company may after some experimenting require everyone who wishes to see the medical professional to use this application.
    After some more time, the company could leverage this to add more and more institutions and control the available "market" of medical professionals.
    It is highly dubious for a single (private) entity to have this much control over medical accessibility.
    
    \item Discrimination:
    Again, families who can't afford the phone and the service for their children would be at a disadvantage because everyone who uses the app would be prioritized and get to a doctor sooner than those outside of this system.
    This would make the society less just because families who can afford this service would get unfairly prioritized.\footnote{In many European countries this is how it works - one can either wait to see a professional or pay for a private one. But that does not mean that it is just.}
    \end{itemize}
\end{enumerate}

%%%%%%%%%%%%%%%%%%%%%%%%%%%%%%%%%%%%%%%%%%%%%%%%%%%%%%%%%%%%%%%%%%%%%%%%

\end{document}
