\documentclass[a4paper, 11pt]{article}
\usepackage[utf8]{inputenc}
\usepackage[T1]{fontenc}
\usepackage[left=2cm,right=5cm,top=2cm,bottom=2cm]{geometry}
\usepackage[english]{babel}
\usepackage{graphicx}
\usepackage{datetime}
\usepackage{enumerate}
\usepackage{xcolor}
\usepackage[onehalfspacing]{setspace}
\setlength{\parindent}{0em}
\setlength{\parskip}{1em}
\newcommand{\stdn}{} \newcommand{\studentname}[1]{\renewcommand{\stdn}{#1}}
\newcommand{\mtrk}{} \newcommand{\matriculationno}[1]{\renewcommand{\mtrk}{#1}}
\newcommand{\isnr}{} \newcommand{\issuenumber}[1]{\renewcommand{\isnr}{#1}}
\newcommand{\makeheader}{\textbf{Ethics for Nerds} Exam \hfill \textbf{Student name:} \stdn \\ \today, \currenttime \hfill \textbf{Matriculation no.:} \mtrk \par \textbf{\Large Issue \isnr}}

\newcommand{\TODO}[1]{\textcolor{red}{TODO: #1}}

\begin{document}
\sffamily

%PLEASE DON'T CHANGE ANYTHING ABOVE THIS POINT

\studentname{Vilém Zouhar} %FILL THIS IN WITH YOUR NAME
\matriculationno{} %FILL THIS IN WITH YOUR MATRICULATION NO.
\issuenumber{2} %FILL THIS IN WITH THE NUMBER OF THE ISSUE
\makeheader %do not delete this line

%%%%%%%%%%%%%%%%%%%%%%%%%% ANSWER THE ISSUE HERE %%%%%%%%%%%%%%%%%%%%%%%%%%%%%%

(Overleaf uses GMT0 while we are in GMT+2)

\begin{enumerate}[(a)]
	\item
	Moral theories and education can still be vital to be able to interpret and understand codes of ethics.
    The codes of ethics have to be justified through morals and they only apply to specific scenarios for professionals.
    In addition, no codes of ethics are probably complete and deep enough and new situations will require further knowledge and theories to extrapolate from the codes of ethics and intuition.
    With time the codes of ethics may become less and less relevant and accurate while moral theories continue to be relevant for longer.
    
    Finally, someone has to write and maintain the codes of ethics.
    Those are professional ethicists who need precisely these theories to write applicable rules.
    
    \item
    The situation he finds himself in is centred around professional conduct.
    
    The preamble clearly states:
    \emph{The Code is designed to inspire and guide the ethical conduct of all computing professionals, including current and aspiring practitioners, instructors, students, influencers, and anyone who uses computing technology in an impactful way.}
    Because John is a computing practitioner and uses computing technology in an impactful way (the impacts are stated in the text) the guide applies to him as well.
    
    \item
    \begin{itemize}
        \item 1.3 \emph{Be honest and trustworthy}\\
        John is certainly not being honest when he systematically withholds information about his troubles. 
        \item 2.1 \emph{Strive to achieve high quality in both the processes and products of professional work.}\\
        The description mentions \emph{transparent communication about the project} which is problematic in the case of John.
        \item 2.2 \emph{Maintain high standards of professional competence, conduct, and ethical practice.}\\
        Although John genuinely strives for high-quality technical standards, he is not communicating his inability to fulfil them. 
        John's ongoing refusal to be honest about his issues is in conflict with the competence to communication.
        \item 2.6 \emph{Perform work only in areas of competence.}\\
        John decided to perform work outside the areas of competence. Though he may not have known this at the beginning, through time he should have gathered enough evidence to make a decision.
    \end{itemize}
    
    \item
        \begin{itemize}
        \item 1.6 \emph{Respect privacy}\\
        The text does not mention any specific stakes related to privacy violations and it is safe to assume that John's action would not affect anything in this regard.
        \item 1.7 \emph{Honor confidentiality.}\\
        Similarly, there is no mention of confidential information and John's actions would not cause any breaches of confidentiality.
    \end{itemize}
    
    \item
    Most of them do not apply as John is not in a leadership role.
    3.6 could be slightly applicable to John because he is working on a presumably existing system though he does not appear to be in a decision making position.
    3.7 seems slightly applicable as well at first though the text mentions that the projects, though containing numerous stakeholders, are not critical or public interest.
    
    \item
    John is clearly breaching principle 1.3 by not disclosing his issues and lack of competence.
    Although he is striving for high-quality work, he also knows that he is unable to provide it. At first, this seems to be in line with 2.1 because he is \emph{striving} though his efforts at \emph{transparent communication} are lacking and therefore he violates this principle as well.
    Even though he tries to maintain high standards of technical competence, he is unable to do so. Consciously he withholds information about his skills which is in conflict with \emph{professional competence - skill in communication}.
    John made a bad decision regarding his suitability for the project. His knowledge of this came only gradually which may have caused the awkward situation in which he finds himself but according to the principle would still be considered rather unethical.
    
    Principle 2.4 does not seem to be in violation with John's actions because he accepts the provided reviews and criticism well.
    
    Overall the principles of the ACM Code of Ethics and Professional Conduct support John's telling his colleagues that he needs much more assistance and disclosing that he is having severe issues.
    Pragmatically even for his own personal well-being, this seems like a good strategy.
    The issue that is preventing him from doing so is probably the immediate upfront cost of being afraid to be ridiculed.
    But this cost will only go higher and he can't keep doing this forever and he is aware of this.
    Therefore it makes rational sense for him to disclose his issues as soon as possible.
\end{enumerate}


%%%%%%%%%%%%%%%%%%%%%%%%%%%%%%%%%%%%%%%%%%%%%%%%%%%%%%%%%%%%%%%%%%%%%%%%

\end{document}
